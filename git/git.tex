\documentclass{programmingnotes}
\begin{document}
\renewcommand{\footrulewidth}{0.4pt}
\fancyhead[LE,LO]{Git -- cheat sheet (page \thepage/\pageref{LastPage})}
\fancyhead[RE,RO]{author: Remigiusz Suwalski, date: \today}
\fancyfoot[RF]{source: \url{https://goo.gl/vFAfEJ}}

\begin{multicols*}{3}
\textbf{Git}, a version control system, tracks changes in files and coordinates work on those files among multiple people.
Other VCS are \emph{Subversion} or \emph{Mercurial}.

\section*{Initialisation} 
\subsection*{\dotfill git clone}
\textbf{git clone <url>} clones an existing repository into new directory.

\subsection*{\dotfill git init}
\textbf{git init} creates an empty Git repository.

\section*{Introducing changes} 
\subsection*{\dotfill git add}
\textbf{git add <file>} stages a file.

\textbf{git add -p <file>} stages a file interactively.

\subsection*{\dotfill git commit}
\textbf{git commit -m} commits staged files with a message.

\textbf{git commit -{}-amend} modifies a commit (and rewrites history!)

\section*{Branches}
\subsection*{\dotfill git branch}
\textbf{git branch}	lists all local branches.

\textbf{git branch -av} lists all local and remote branches.

\textbf{git branch <branch>} creates a new branch.

\textbf{git branch -d <branch>} deletes a branch.

\textbf{git branch -{}-contains <commit>} lists branches with given commit.

\subsection*{\dotfill git checkout}
\textbf{git checkout <branch>} switches to an existing branch and updates working directory.

\textbf{git checkout -b <branch>} creates a new branch and switches to it.

\textbf{git checkout -{}-orphan <branch>} ?.

\subsection*{\dotfill git chery-pick}
\textbf{git cherry-pick <commit>} adds a commit on top of current branch.

\subsection*{\dotfill git merge}
\textbf{git merge <branch>} merges all changes into current branch.
Combined with \textbf{fetch} almost equals \textbf{pull}.

\textbf{git merge -{}-abort} ?.

\subsection*{\dotfill git rebase}
\textbf{git rebase <branch>} rebases: reapplies commits on top of another base tip.

\textbf{git rebase -i HEAD\textasciitilde<number>} rebases interactively.

\textbf{git rebase abort}

\subsection*{\dotfill git remote}
\textbf{git remote -v} lists tracked remote repositories.

\textbf{git remote show <remote>} shows information about remote repository.

\textbf{git remote add <name> <url>} adds a new remote repository.

\section*{Observing changes}
\subsection*{\dotfill git blame}
\textbf{git blame <file>} shows what revision and author last modified each line of a file.

\subsection*{\dotfill git diff} 
\textbf{git diff} lists unstaged changes.

\textbf{git diff <commit>} lists changes between workspace and the commit.
\textbf{git diff <commit> <commit>} shows changes between two commits.
Above commands work with \textbf{<branch>} in place of \textbf{<commit>} too.

\textbf{git diff -{}-cached} shows changes to staged files.

\subsection*{\dotfill git log}
\textbf{git log} shows full version history.

\textbf{git log -p <file>} shows file's change history.

\textbf{git log -{}-follow <file>} includes renames.

\textbf{git log -{}-oneline} shows compact history.

\textbf{git log -{}-all -{}-decorate -{}-oneline -{}-graph}

\subsection*{\dotfill git show}
\textbf{git show <commit>:<file>} shows contents of a file.

\subsection*{\dotfill git status}
\textbf{git status} lists new and modified files.

\section*{Undoing changes} 
\subsection*{\dotfill git reset}
\textbf{git reset <file>} unstages file but keeps the changes.

\textbf{git reset -{}-hard <file>} throws away all local changes.

\subsection*{\dotfill git revert}
\textbf{git revert <commit>} undoes a commit.

\section*{(Re)moving} 
\textbf{git mv} moves.

\textbf{git rm} removes.

\section*{Syncronising} 

\subsection*{\dotfill git fetch}
\textbf{git fetch <remote>} gets the latest changes from origin without merge.

\subsection*{\dotfill git pull}
\textbf{git pull <remote> <branch>} fetches latest changes and merges

\textbf{git pull -{}--rebase} fetches latest changes and rebases.

\subsection*{\dotfill git push}
\textbf{git push <remote> <branch>} pushes local changes to origin

\section*{Various}
\subsection*{\dotfill git tag}
\textbf{git tag <tagname>} tags the current commit.
\end{multicols*}
\end{document}